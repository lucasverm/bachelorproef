%---------- Inleiding ---------------------------------------------------------

\section*{Introductie} % The \section*{} command stops section numbering
\label{sec:introductie}

In de 21e eeuw heerst er nog steeds een grote ongelijkheid tussen de verschillende werelddelen, ondanks het feit dat we er als wereld in zijn geheel constant op vooruit gaan. In de meeste landen van Europa en Noord-Amerika stellen de burgers het over het algemeen goed tot zeer goed. Er is voldoende voedsel, er is toegang tot goede gezondheidszorg, er is een behoorlijk uitgebouwd onderwijssysteem, en ga zo maar door. 

Dat is echter niet overal het geval. Veel landen hebben te kampen met een snelle bevolkingsgroei, met hoge sterftecijfers en een grote kloof tussen arm en rijk. Veelal hebben deze landen een duidelijke ontwikkelingsachterstand.

Er werd al veel onderzoek gedaan naar hoe ICT invloed kan hebben op de ontwikkeling van een land. Dit onderzoek zal zich toespitsen op de invloed van ICT in het onderwijs in Peru. Het heeft meer specifiek het doel om de invloed van ICT op het niveau van het onderwijs in Peru te analyseren en in kaart te brengen, en te onderzoeken hoe het de ontwikkeling van dit land kan versnellen.

\subsection{Onderzoekvragen}
$Onderzoeksvraag 1$: Wat zijn de minimale ICT competenties nodig om in een rendabel bedrijf in een onderontwikkeld land te kunnen werken?

$Onderzoeksvraag 2$: Hoe kan de introductie van ICT in het onderwijs de ontwikkeling van het land kan verbeteren?

%Wat zijn de minimale ICT competenties nodig om in een rendabel multinational in een onderontwikkeld land te kunnen werken?


%---------- Stand van zaken hehe ---------------------------------------------------

\section{Stand van zaken}
\label{sec:state-of-the-art}
Ontwikkelingslanden hebben economische, technologische, wetenschappelijke en medische achterstand ten opzicht van ontwikkelde landen. De Verenigde naties deelt alle landen van de wereld op in drie groepen. Daaruit blijkt dat er 43 ontwikkelde landen zijn, 17 landen die de overgang aan het maken zijn van onderontwikkeld naar ontwikkeld land, maar dat er maar liefst 127 onderontwikkelde landen zijn. Peru behoort tot de laatste groep. \autocite{unitednations2019} 

Sinds de jaren ’90 heeft informatie communicatie technologie (ICT) veel invloed op onze samenlevingen. Op vele vlakken evolueren we heel snel, onder meer door middel van ICT. Door de inbreng van ICT in onze samenlevingen veranderde ons leven en konden we zaken efficiënter doen. In Peru heeft op dit moment 53\% van de bevolking toegang tot het internet, tegenover 89\% in België. \autocite{itu2018} Ook op het vlak van onderwijs kan er geëvolueerd worden, om zo de kwaliteit van het onderwijs te verbeteren.

In 2001 werd het Huascaran Project opgestart in Peru. Het doel van het project was om landelijke netwerken te ontwikkelen, en ICT te implementeren in openbare scholen, meer bepaalt door hen uit te rusten met een server en met toegang tot het internet, om op die manier de kwaliteit van het onderwijs te verbeteren. Er werden ongeveer 15.000 computers verspreid over scholen in het hele land en 55.000 leraren werden opgeleid. Echter, het project werd nooit geëvalueerd, en eindigde in 2007. \autocite{salas-pilco2014}

In 2007 werd het (wereldwijde) ’One Laptop Per Child’ project geïntroduceerd in Peru. Het was een project met als doel elk kind op school te voorzien van een laptop, om zo het onderwijs naar een hoger niveau te tillen. Nicholas Negroponte, de vader van het project, had zich tot doel gesteld om armoede door middel van computers terug te dringen. Hij ontwierp een apparaat om de Derde Wereld te helpen. Dat project faalde, door tal van projectmatige fouten. \autocite{Wooster2018} Of het project nu echt kinderen ‘slimmer’ maakte is er onenigheid: een studie van de Inter-Amerikaanse Ontwikkelingsbank \autocite{Severin2012} wees uit dat Peruaanse kinderen met laptops zes maanden voorsprong hadden op hun leeftijdsgenoten op vlak van logisch redeneren en hun verbaal vermogen, maar het onderzoek kon geen verbeteringen vinden op het gebied van wiskunde, taal of leesgewoonten. \autocite{Murhpy2012}
 

%---------- Methodologie ------------------------------------------------------
\section{Methodologie}
\label{sec:methodologie}
Desk research zal gebeuren door de literatuur die al over ICT en educatie in Peru werd geschreven, op te lijsten en te analyseren. Eigen field research zal geburen door ter plaatse te gaan kijken in welke toestand het invoeren van ICT in het onderwijs zich momenteel bevindt, en via bevraging van gebruikers zal gekeken worden wat de bevindingen van de Peruvianen zelf zijn.
 
 %beschrijf je hoe je van plan bent het onderzoek te voeren. Welke onderzoekstechniek ga je toepassen om elk van je onderzoeksvragen te beantwoorden? Gebruik je hiervoor experimenten, vragenlijsten, simulaties? Je beschrijft ook al welke tools je denkt hiervoor te gebruiken of te ontwikkelen.

%---------- Verwachte resultaten ----------------------------------------------
\section{Verwachte resultaten}
\label{sec:verwachte_resultaten}
Er wordt verwacht dat er in publieke scholen in Peru nog onvoldoende aandacht is voor ICT. Er wordt naar verwachting nog onvoldoende gebruik gemaakt van ICT tijdens leermomenten op school. Veelal wordt er gebruik gemaakt van traditionele methoden.
%Hier beschrijf je welke resultaten je verwacht. Als je metingen en simulaties uitvoert, kan je hier al mock-ups maken van de grafieken samen met de verwachte conclusies. Benoem zeker al je assen en de stukken van de grafiek die je gaat gebruiken. Dit zorgt ervoor dat je concreet weet hoe je je data gaat moeten structureren.

%---------- Verwachte conclusies ----------------------------------------------
\section{Verwachte conclusies}
\label{sec:verwachte_conclusies}
Vermoedelijk komt dit omdat grote delen van Peru nog onderontwikkeld zijn. Hierdoor heeft het land geen budget om te investeren in ICT.
%Hier beschrijf je wat je verwacht uit je onderzoek, met de motivatie waarom. Het is \textbf{niet} erg indien uit je onderzoek andere resultaten en conclusies vloeien dan dat je hier beschrijft: het is dan juist interessant om te onderzoeken waarom jouw hypothesen niet overeenkomen met de resultaten.

