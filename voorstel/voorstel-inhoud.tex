%---------- Inleiding ---------------------------------------------------------

\section{Introductie} % The \section*{} command stops section numbering
\label{sec:introductie}

In de wereld heerst er nog steeds veel verdeeldheid ondanks we er op vele vlakken op vooruit gaan. In Europa en Amerika stellen de burgers het goed. Er is genoeg voedsel, toegang tot goede gezondheidszorg, een uitgebouwd onderwijssysteem, en ga zo maar door. Dat is niet overal het geval. Veel landen hebben te kampen met snelle bevolkingsgroei, hoge sterftecijfers en een grote kloof tussen arm en rijk. Veelal hebben deze landen een ontwikkelingsachterstand. Ze hebben geen hogere graad van industrialisatie bereikt in verhouding tot ontwikkelde landen. 
%https://www.un.org/development/desa/dpad/wp-content/uploads/sites/45/WESP2019_BOOK-web.pdf

Er werd al veel onderzoek gedaan naar hoe ICT invloed kan hebben op de ontwikkeling van een land. Dit onderzoek valt onder de algemene term "ICT4Development". Dit specifiek onderzoek zal gaan over hoe ICT Peru beter kan doen ontwikkelen. 

%---------- Stand van zaken hehe ---------------------------------------------------

\section{Stand van zaken}
\label{sec:state-of-the-art}
Ontwikkelingslanden hebben economische, technologische, wetenschappelijke en medische achterstanden. De Verenigde naties deelt alle landen van de wereld op in 3 tabellen. Daaruit blijkt dat er 43 ontwikkelde landen, 17 landen de overgang aan het maken zijn tussen ontwikkeld en onderontwikkeld, en 127 onderontwikkelde landen zijn. \autocite{Nations2019} Er blijkt dus dat er meer landen zijn die onderontwikkeld zijn. Peru behoort bij de laatste groep van de onderontwikkelde landen. 

ICT kan een positieve invloed hebben op ontwikkeling, als het op de juist manier wordt toegepast. Op dit moment heeft 89 van de Belgische bevolking toegang tot internet, tegenover 53 in Peru. \cite{ITU2018} Hiermee zal zeker rekening moeten gehouden worden tijden het onderzoek. Vorig jaar pompte Belgi\"e iets meer dan 600 miljoen euro in ontwikkelingssamenwerking, waarvan  3 miljoen naar Peru ging. 

%https://data.worldbank.org/indicator/IT.NET.USER.ZS?end=2018&locations=BE&name_desc=false&start=1960&view=chart

Hier beschrijf je de \emph{state-of-the-art} rondom je gekozen onderzoeksdomein. Dit kan bijvoorbeeld een literatuurstudie zijn. Je mag de titel van deze sectie ook aanpassen (literatuurstudie, stand van zaken, enz.). Zijn er al gelijkaardige onderzoeken gevoerd? Wat concluderen ze? Wat is het verschil met jouw onderzoek? Wat is de relevantie met jouw onderzoek?

Verwijs bij elke introductie van een term of bewering over het domein naar de vakliteratuur, bijvoorbeeld~\autocite{Doll1954}! Denk zeker goed na welke werken je refereert en waarom.

% Voor literatuurverwijzingen zijn er twee belangrijke commando's:
% \autocite{KEY} => (Auteur, jaartal) Gebruik dit als de naam van de auteur
%   geen onderdeel is van de zin.
% \textcite{KEY} => Auteur (jaartal)  Gebruik dit als de auteursnaam wel een
%   functie heeft in de zin (bv. ``Uit onderzoek door Doll & Hill (1954) bleek
%   ...'')

Je mag gerust gebruik maken van subsecties in dit onderdeel.

%---------- Methodologie ------------------------------------------------------
\section{Methodologie}
\label{sec:methodologie}
Ik zal bekijken wat er al onderzocht werd over ICT4Development. 
Ik zal onderzoeken in hoeverre Peru toegang heeft tot ICT en het internet tegenover de toegang tot deze media in een ontwikkeld land (Belgi\"e). Dit zal door middel van een vragenlijst onderzocht worden, zo zullen de verschillen tussen een ontwikkeld en onontwikkeld land bloot komen te liggen. Ook zal de huidige stand van zaken omtrent economie technologie, wetenschap en gezondheidszorg onderzocht, en vergeleken worden. Verder zal ik onderzoeken waarin Peru tekort schiet. Waar hebben inwoners nood aan. Hoe kan het de ontwikkeling van het land stimuleren.
 
 Ik hoop met een aantal concrete tips naar buiten te kunnen komen over hoe Peru beter kan ontwikkelen via ICT.
 
 %beschrijf je hoe je van plan bent het onderzoek te voeren. Welke onderzoekstechniek ga je toepassen om elk van je onderzoeksvragen te beantwoorden? Gebruik je hiervoor experimenten, vragenlijsten, simulaties? Je beschrijft ook al welke tools je denkt hiervoor te gebruiken of te ontwikkelen.

%---------- Verwachte resultaten ----------------------------------------------
\section{Verwachte resultaten}
\label{sec:verwachte_resultaten}

Hier beschrijf je welke resultaten je verwacht. Als je metingen en simulaties uitvoert, kan je hier al mock-ups maken van de grafieken samen met de verwachte conclusies. Benoem zeker al je assen en de stukken van de grafiek die je gaat gebruiken. Dit zorgt ervoor dat je concreet weet hoe je je data gaat moeten structureren.

%---------- Verwachte conclusies ----------------------------------------------
\section{Verwachte conclusies}
\label{sec:verwachte_conclusies}

Hier beschrijf je wat je verwacht uit je onderzoek, met de motivatie waarom. Het is \textbf{niet} erg indien uit je onderzoek andere resultaten en conclusies vloeien dan dat je hier beschrijft: het is dan juist interessant om te onderzoeken waarom jouw hypothesen niet overeenkomen met de resultaten.

