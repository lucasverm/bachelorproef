%---------- Inleiding ---------------------------------------------------------

%DESK RESEARCH AND FIELD RESEARCH
%1/2 kwlitatief/kwantit

\section{Introductie} % The \section*{} command stops section numbering
\label{sec:introductie}

In de wereld heerst nog steeds  verdeeldheid ondanks we er op vele vlakken op vooruit gaan. In de meeste landen van  Europa en Amerika stellen de burgers het goed. Er is genoeg voedsel, toegang tot goede gezondheidszorg, een uitgebouwd onderwijssysteem, en ga zo maar door. Dat is niet overal het geval. Veel landen hebben te kampen met snelle bevolkingsgroei, hoge sterftecijfers en een grote kloof tussen arm en rijk. Veelal hebben deze landen een ontwikkelingsachterstand. 
%https://www.un.org/development/desa/dpad/wp-content/uploads/sites/45/WESP2019_BOOK-web.pdf

Er werd al veel onderzoek gedaan naar hoe ICT invloed kan hebben op de ontwikkeling van een land. Dit onderzoek zal zich specifieren op ICT in het onderwijs. Het heeft als doel het onderzoeken van de invloed van ICT op het niveau van het onderwijs in Peru te onderzoeken, en te onderzoeken hoe dit kan verbeterd worden. 

%---------- Stand van zaken hehe ---------------------------------------------------

\section{Stand van zaken}
\label{sec:state-of-the-art}
Ontwikkelingslanden hebben economische, technologische, wetenschappelijke en medische achterstanden tegenover ontwikkelde landen. De Verenigde naties deelt alle landen van de wereld op in 3 tabellen. Daaruit blijkt dat er 43 ontwikkelde landen, 17 landen de overgang aan het maken zijn tussen ontwikkeld en onderontwikkeld, en 127 onderontwikkelde landen zijn. \autocite{Nations2019} Peru behoort bij de laatste groep van de onderontwikkelde landen. 

Sinds de jaren '90 heeft informatie communicatie technologie (ICT) veel invloed op onze samenlevingen. Op vele vlakken evolueren we heel snel door middel van ICT. Door deze nieuwe trend veranderde ons leven, en kunnen we zaken  effici\"enter doen. In Peru heeft op dit moment 53\% van de bevolking toegang tot het internet, tegenover 89\% in Belgi\"e. \autocite{ITU2018} Ook op vlak van onderwijs kan er ge\"evolueerd worden, om zo de kwaliteit van het onderwijs te verbeteren.

In 2001 werd het Huascaran Project opgestart in Peru. Het doel van het project was om landelijke netwerken te ontwikkelen, implementeren en evalueren voor openbare scholen en het uitrusten van deze scholen met een server en toegang tot internet. Om op die manier de kwaliteit van het onderwijs te verbeteren. Er werden ongeveer 15.000 computers verspreid tussen scholen en 55.000 leraren werden opgeleid. Echter, het project werd nooit ge\"evalueerd, en eindigde in 2007.  \autocite{SALAS-PILCO2014}

In 2007 startte werd het wereldwijde One Laptop Per Child gen\"introduceerd in Peru. Dat was een project met als doel, elk kind op school te voorzien van een laptop, om zo het onderwijs naar een hoger niveau te helpen.
 

%---------- Methodologie ------------------------------------------------------
\section{Methodologie}
\label{sec:methodologie}
Desk research zal gebeuren door de literatuur die al over ICT en educatie in Peru werd geschreven te analysen. Het verdere field research zal plaatsvinden door ter plaatse te gaan, en via vragenlijsten zal na worden gegaan wat de bevindingen van de Peruvianen zelf zijn.
 
 %beschrijf je hoe je van plan bent het onderzoek te voeren. Welke onderzoekstechniek ga je toepassen om elk van je onderzoeksvragen te beantwoorden? Gebruik je hiervoor experimenten, vragenlijsten, simulaties? Je beschrijft ook al welke tools je denkt hiervoor te gebruiken of te ontwikkelen.

%---------- Verwachte resultaten ----------------------------------------------
\section{Verwachte resultaten}
\label{sec:verwachte_resultaten}
Er wordt verwacht dat er in de publieke scholen in Peru nog  geen aandacht is voor ICT. Er zou dus weinig of geen gebruik worden gemaakt van ICT tijdens het onderwijzen. Veelal wordt er gebruik gemaakt van traditionele methoden.
%Hier beschrijf je welke resultaten je verwacht. Als je metingen en simulaties uitvoert, kan je hier al mock-ups maken van de grafieken samen met de verwachte conclusies. Benoem zeker al je assen en de stukken van de grafiek die je gaat gebruiken. Dit zorgt ervoor dat je concreet weet hoe je je data gaat moeten structureren.

%---------- Verwachte conclusies ----------------------------------------------
\section{Verwachte conclusies}
\label{sec:verwachte_conclusies}
Vermoedelijk zou dit kunnen komen omdat Peru een onderontwikkeld land is. Hierdoor heeft het land geen budget om te investeren in ICT.
%Hier beschrijf je wat je verwacht uit je onderzoek, met de motivatie waarom. Het is \textbf{niet} erg indien uit je onderzoek andere resultaten en conclusies vloeien dan dat je hier beschrijft: het is dan juist interessant om te onderzoeken waarom jouw hypothesen niet overeenkomen met de resultaten.

