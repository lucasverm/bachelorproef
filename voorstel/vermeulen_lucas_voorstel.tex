%==============================================================================
% Sjabloon onderzoeksvoorstel bachelorproef
%==============================================================================
% Gebaseerd op LaTeX-sjabloon ‘Stylish Article’ (zie voorstel.cls)
% Auteur: Jens Buysse, Bert Van Vreckem
%
% Compileren in TeXstudio:
%
% - Zorg dat Biber de bibliografie compileert (en niet Biblatex)
%   Options > Configure > Build > Default Bibliography Tool: "txs:///biber"
% - F5 om te compileren en het resultaat te bekijken.
% - Als de bibliografie niet zichtbaar is, probeer dan F5 - F8 - F5
%   Met F8 compileer je de bibliografie apart.
%
% Als je JabRef gebruikt voor het bijhouden van de bibliografie, zorg dan
% dat je in ``biblatex''-modus opslaat: File > Switch to BibLaTeX mode.

\documentclass{voorstel}

\usepackage{lipsum}

%------------------------------------------------------------------------------
% Metadata over het voorstel
%------------------------------------------------------------------------------

%---------- Titel & auteur ----------------------------------------------------

\PaperTitle{Wat is ICT4Development, en hoe kan het de levenskwaliteit in Peru verbeteren?}
\PaperType{Onderzoeksvoorstel Bachelorproef 2019-2020} % Type document

% TODO: vul je eigen naam in als auteur, geef ook je emailadres mee!
\Authors{Lucas Vermeulen\textsuperscript{1}}
\CoPromotor{Ellen Bosch\textsuperscript{2} (A\~na\~nau)}
\affiliation{\textbf{Contact:}
  \textsuperscript{1} \href{mailto:lucas.vermeulen@student.hogent.be}{lucas.vermeulen@student.hogent.be};
  \textsuperscript{2} \href{mailto:info@ananau.org}{info@ananau.org};
}

%---------- Abstract ----------------------------------------------------------

\Abstract{Hier schrijf je de samenvatting van je voorstel, als een doorlopende tekst van één paragraaf. Wat hier zeker in moet vermeld worden: \textbf{Context} (Waarom is dit werk belangrijk?); \textbf{Nood} (Waarom moet dit onderzocht worden?); \textbf{Taak} (Wat ga je (ongeveer) doen?); \textbf{Object} (Wat staat in dit document geschreven?); \textbf{Resultaat} (Wat verwacht je van je onderzoek?); \textbf{Conclusie} (Wat verwacht je van van de conclusies?); \textbf{Perspectief} (Wat zegt de toekomst voor dit werk?).

Bij de sleutelwoorden geef je het onderzoeksdomein, samen met andere sleutelwoorden die je werk beschrijven.

Vergeet ook niet je co-promotor op te geven.
}

%---------- Onderzoeksdomein en sleutelwoorden --------------------------------
% TODO: Sleutelwoorden:
%
% Het eerste sleutelwoord beschrijft het onderzoeksdomein. Je kan kiezen uit
% deze lijst:
%
% - Mobiele applicatieontwikkeling
% - Webapplicatieontwikkeling
% - Applicatieontwikkeling (andere)
% - Systeembeheer
% - Netwerkbeheer
% - Mainframe
% - E-business
% - Databanken en big data
% - Machineleertechnieken en kunstmatige intelligentie
% - Andere (specifieer)
%
% De andere sleutelwoorden zijn vrij te kiezen

\Keywords{ICT4Development. ICT4D --- Ontwikkeling --- Peru --- } % Keywords
\newcommand{\keywordname}{Sleutelwoorden} % Defines the keywords heading name

%---------- Titel, inhoud -----------------------------------------------------

\begin{document}

\flushbottom % Makes all text pages the same height
\maketitle % Print the title and abstract box\tableofcontents % Print the contents section
\thispagestyle{empty} % Removes page numbering from the first page

%------------------------------------------------------------------------------
% Hoofdtekst
%------------------------------------------------------------------------------

% De hoofdtekst van het voorstel zit in een apart bestand, zodat het makkelijk
% kan opgenomen worden in de bijlagen van de bachelorproef zelf.
%---------- Inleiding ---------------------------------------------------------

\section{Introductie} % The \section*{} command stops section numbering
\label{sec:introductie}

In de wereld heerst nog steeds  verdeeldheid ondanks we er op vele vlakken op vooruit gaan. In de meeste landen van  Europa en Noord-Amerika stellen de burgers het goed. Er is genoeg voedsel, toegang tot goede gezondheidszorg, een uitgebouwd onderwijssysteem, en ga zo maar door. Dat is niet overal het geval. Veel landen hebben te kampen met snelle bevolkingsgroei, hoge sterftecijfers en een grote kloof tussen arm en rijk. Veelal hebben deze landen een ontwikkelingsachterstand. 
%https://www.un.org/development/desa/dpad/wp-content/uploads/sites/45/WESP2019_BOOK-web.pdf

Er werd al veel onderzoek gedaan naar hoe ICT invloed kan hebben op de ontwikkeling van een land. Dit onderzoek zal zich specificeren op ICT in het onderwijs in Peru. Het heeft als doel de invloed van ICT op het niveau van het onderwijs in Peru te analyseren, en te onderzoeken hoe dit kan verbeterd worden. 

%---------- Stand van zaken hehe ---------------------------------------------------

\section{Stand van zaken}
\label{sec:state-of-the-art}
Ontwikkelingslanden hebben economische, technologische, wetenschappelijke en medische achterstanden tegenover ontwikkelde landen. De Verenigde naties deelt alle landen van de wereld op in 3 tabellen. Daaruit blijkt dat er 43 ontwikkelde landen, 17 landen de overgang aan het maken zijn tussen ontwikkeld en onderontwikkeld, en 127 onderontwikkelde landen zijn. \autocite{unitednations2019} Peru behoort bij de laatste groep van de onderontwikkelde landen. 

Sinds de jaren '90 heeft informatie communicatie technologie (ICT) veel invloed op onze samenlevingen. Op vele vlakken evolueren we heel snel door middel van ICT. Door deze nieuwe trend veranderde ons leven, en kunnen we zaken  effici\"enter doen. In Peru heeft op dit moment 53\% van de bevolking toegang tot het internet, tegenover 89\% in Belgi\"e. \autocite{itu2018} Ook op vlak van onderwijs kan er ge\"evolueerd worden, om zo de kwaliteit van het onderwijs te verbeteren.

In 2001 werd het Huascaran Project opgestart in Peru. Het doel van het project was om landelijke netwerken te ontwikkelen, implementeren en evalueren voor openbare scholen en het uitrusten van deze scholen met een server en toegang tot internet. Om op die manier de kwaliteit van het onderwijs te verbeteren. Er werden ongeveer 15.000 computers verspreid over scholen in het hele lang en 55.000 leraren werden opgeleid. Echter, het project werd nooit ge\"evalueerd, en eindigde in 2007.  \autocite{salas-pilco2014}

In 2007 startte het wereldwijde 'One Laptop Per Child' gen\"introduceerd in Peru. Dat was een project met als doel, elk kind op school te voorzien van een laptop, om zo het onderwijs naar een hoger niveau te tillen. Nicholas Negroponte, de vader van het project, was gericht op het beëindigen van armoede door middel van computers. Hij ontwierp een apparaat om de Derde Wereld te helpen. Dat project faalde, door vele projectmatige fouten. \autocite{Wooster2018} Over of het project nu echt de kinderen slimmer maakte, is er onenigheid: een studie \autocite{Severin2012} van de Inter-Amerikaanse Ontwikkelingsbank wees uit dat Peruaanse kinderen met laptops zes maanden voorsprong hadden op hun leeftijdsgenoten op vlak van logisch redeneren en hun verbaal vermogen, maar het onderzoek kon  geen verbeteringen vinden op gebieden als wiskunde, taal of leesgewoonten. \autocite{Murhpy2012}
 

%---------- Methodologie ------------------------------------------------------
\section{Methodologie}
\label{sec:methodologie}
De desk research zal gebeuren door de literatuur die al over ICT en educatie in Peru werd geschreven te analysen. Het verdere field research zal plaatsvinden door ter plaatse te gaan, en via vragenlijsten zal gekeken worden wat de bevindingen van de Peruvianen zelf zijn.
 
 %beschrijf je hoe je van plan bent het onderzoek te voeren. Welke onderzoekstechniek ga je toepassen om elk van je onderzoeksvragen te beantwoorden? Gebruik je hiervoor experimenten, vragenlijsten, simulaties? Je beschrijft ook al welke tools je denkt hiervoor te gebruiken of te ontwikkelen.

%---------- Verwachte resultaten ----------------------------------------------
\section{Verwachte resultaten}
\label{sec:verwachte_resultaten}
Er wordt verwacht dat er in de publieke scholen in Peru nog  onvoldoende aandacht is voor ICT. Er zou dus weinig of geen gebruik worden gemaakt van ICT tijdens het onderwijzen. Veelal wordt er gebruik gemaakt van traditionele methoden.
%Hier beschrijf je welke resultaten je verwacht. Als je metingen en simulaties uitvoert, kan je hier al mock-ups maken van de grafieken samen met de verwachte conclusies. Benoem zeker al je assen en de stukken van de grafiek die je gaat gebruiken. Dit zorgt ervoor dat je concreet weet hoe je je data gaat moeten structureren.

%---------- Verwachte conclusies ----------------------------------------------
\section{Verwachte conclusies}
\label{sec:verwachte_conclusies}
Vermoedelijk zou dit kunnen komen omdat Peru een onderontwikkeld land is. Hierdoor heeft het land geen budget om te investeren in ICT.
%Hier beschrijf je wat je verwacht uit je onderzoek, met de motivatie waarom. Het is \textbf{niet} erg indien uit je onderzoek andere resultaten en conclusies vloeien dan dat je hier beschrijft: het is dan juist interessant om te onderzoeken waarom jouw hypothesen niet overeenkomen met de resultaten.



%------------------------------------------------------------------------------
% Referentielijst
%------------------------------------------------------------------------------
% TODO: de gerefereerde werken moeten in BibTeX-bestand ``voorstel.bib''
% voorkomen. Gebruik JabRef om je bibliografie bij te houden en vergeet niet
% om compatibiliteit met Biber/BibLaTeX aan te zetten (File > Switch to
% BibLaTeX mode)

\phantomsection
\printbibliography[heading=bibintoc]

\end{document}
