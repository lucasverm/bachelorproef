%%=============================================================================
%% InterviewSadith
%%=============================================================================

\chapter{Interview met Erick Paez Montesinos, leerkracht}
\label{ch:interviewMetEric}

%% TODO: Hoe ben je te werk gegaan? Verdeel je onderzoek in grote fasen, en
%% licht in elke fase toe welke stappen je gevolgd hebt. Verantwoord waarom je
%% op deze manier te werk gegaan bent. Je moet kunnen aantonen dat je de best
%% mogelijke manier toegepast hebt om een antwoord te vinden op de
%% onderzoeksvraag.
\section{Inleiding}
Erick Paez Montesinos werkte gedurende 9 jaar voor het Peruviaanse ministerie van onderwijs. Daarnaast gaf hij les in verschillende onderwijs instituten en werkte hij in een monitoraat dat andere leerkrachten op pedagogische vlak bijschoolt en begeleid. Op dit moment geeft hij geen les meer, omdat hij een andere richting uitging, maar vorig schooljaar gaf hij les in educación primaria. Dat is het Peruviaanse basis onderwijs, waar kinderen tussen 6 en 11 jaar onder vallen. \autocite{Nuffic2015} Ook gaf hij les in educación initial, educación secundaria en educación superior. Hij is iemand met veel ervaring binnen het onderwijs en kan ons relevante informatie bezorgen voor dit onderzoek.

 %Maestro Titular de Primaria

\section{Interview}

\textbf{In welke school werkte je?}

\textit{Eric:} Ik werkte in een staats school in Apurímac. Dat is niet in Cusco, maar een regio daarbuiten. Daar werkte ik in instituut 50640, genaamd Sagrado corazon de Jesus.
%http://www.dreapurimac.gob.pe/inicio/images/ARCHIVOS_2019/CD-19/CD-2019-UGEL-COTABAMBAS.pdf

\textbf{Hoeveel leerlingen zitten er gemiddeld in de klassen waarin u lesgaf?}

\textit{Eric:} Er zitten normaal gezien 20 tot 25 kinderen in elke klas. Op de hele school waren er 11 klassen, dus er waren ongeveer 220 tot 250 leerlingen op de school waar ik les gaf. %5:06

\textbf{Had je op jouw school computers ter beschikking? Welke apparatuur was er? Zijn er problemen met deze apparatuur?}

\textit{Eric:} Er was 1 computerklas die gebruikt kon worden door leerkrachten met hun leerlingen. Leerkrachten konden het lokaal boeken en de computers in hun les verwerken en gebruiken. Deze computers werden specifiek gebruikt voor informatica lessen. Verder had de school ook laptops die door de overheid betaald werden. Deze laptops konden door klassen uitgeleend worden, en werden gebruikt als hulpmiddel tijdens andere lessen, zoals wiskunde of taal. De laptops hebben minder rekenkracht en zijn alleen voorzien basis functionaliteiten. Ze zijn gemaakt om educatieve programma's op te gebruiken en informatie te raadplegen op het internet. Het zijn wit-groene laptops van het merk XO. De school had ongeveer 70 laptops van dit type. Op dit moment worden ze minder en minder gebruikt, omdat ze verouderd zijn.

\textbf{Had jouw school een ICT leerkracht?}

\textit{Eric:} Ja, onze school had een ICT leerkracht. Hij was verantwoordelijk voor het controleren, en organiseren van alles dat met ICT te maken had. Normaal gezien worden deze leerkrachten geselecteerd en opgeleid door de educatieve koepel van het onderwijs district. Ik denk dat ongeveer 85\% van de publieke scholen in Peru een ICT leerkracht heeft. Bij ons op school was dit iemand die extra lessen had gevolgd, en zich had bijgeschoold tot leerkracht informatica. In andere scholen zijn er ook ICT leerkrachten die effectief informatica gestudeerd hebben.

\textbf{In hoeverre kunnen leerkrachten die geen ICT geven als vak, om met ICT?}

\textit{Eric:} Er wordt verwacht van alle leerkrachten dat ze een basisniveau ICT hebben. Echter komt het vaak voor dat leerkrachten met veel ervaring niet goed overweg kunnen met computers. 

\textbf{In hoeverre kunnen leekrachten die geen ICT geven als vak, om met ICT?}

\textit{Eric:} Er wordt verwacht van alle leerkrachten dat ze een basisniveau ICT hebben. Echter komt het vaak voor dat leerkrachten met veel ervaring niet goed overweg kunnen met computers. 

\textbf{Als leerlingen informatica les krijgen, wat leren ze dan?}

\textit{Eric:} Ze leren niet echt specifiek dingen maar vooral het basis gebruik van de computer. Aan en uit zetten, iets op zoeken op het internet, .. Maar echt bijvoorbeeld in Excel leren werken, of programmeren leren ze niet.

\textbf{Krijgen leerlingen huiswerk op de computer?}

\textit{Eric:} Ja, als kinderen huiswerk krijgen is dat meestal op papier, en gebruiken ze hun laptops als hulp middel. In kleine gemeenschappen, waar de mensen meestal arm zijn, en dus thuis geen computers hebben, zijn er scholen waarop de kinderen in educación primaria hun laptops die ze op school krijgen mee naar huis mogen nemen. Daar kunnen ze via satelliet verbindingen, die gratis zijn, filmpjes kijken, dingen opzoeken en educatieve spelletjes spelen. 

\textbf{Worden de laptops dan niet gestolen? Wordt er geen misbruik gemaakt van het systeem?}

\textit{Eric:} Neen, als dat gebeurd zal de school altijd uitvoerig onderzoeken wat er gebeurde. Dit gaat om educatieve laptops, dus echt interesse is er niet naar. Een leerling kan zijn laptop dus niet verkopen, want niemand zal ze kopen. 

\textbf{Zijn er problemen met de laptops die gebruikt worden?}

\textit{Eric:} Ja, eigenlijk wel. De laptops beginnen te verouderen, daarom worden ze minder en minder gebruikt. Hun schermen zijn ook niet van de beste kwaliteit en ze hebben niet genoeg geheugen capaciteit. Ook zijn de scharnieren niet sterk genoeg, en kunnen ze breken.

\textbf{Hoe komt het dat Peru niet verder staat op vlak van informatica, wat liep er fout in het verleden?}

\textit{Eric:} Het probleem is volgens mij groter dan informatica alleen. Vaak is de toegang tot internet beperkt in kleine gemeenschappen. Deze mensen kunnen niet op internet, en hebben ook geen financiële middelen om een computer te kunnen aankopen. Volgens mij liggen economische problemen aan de basis van de ICT problemen. Er is geen geld voor computers. Niet bij de overheid, maar vaak ook niet bij de gezinnen zelf. Ik denk dat NGO's kunnen helpen om deze problematieken de wereld uit te helpen om elke school te voorzien van informatica. Ik denk dat hier in Peru er een aantal goede NGO's zijn die ons een duw in de rug gaven, en dat dit ook zou kunnen op vlak van informatica. De overheid probeert hun best te doen, om ons vooruit te helpen, maar meestal is er onvoldoende budget, omdat informatica niet prioritair is. Ik denk dat door de recente uitbraak van het covid-19 virus hun ogen wel zijn open gegaan, dat blijk ook want de overheid kocht recent veel computers aan om uit te delen aan kinderen, om vanop afstand les te kunnen volgen. Ik denk dat dit misschien veel zal verbeteren aan deze situatie. De staatsscholen kunnen niet zelf instaan voor de aankoop van hun computers, omdat dat dat voor hen gewoon te duur is, en omdat ze hier niet genoeg budget voor krijgen van de overheid. Nogmaals, ik denk dat de steun van NGO's veel zou kunnen betekenen.

\subsection{Besluit}

