\chapter{Interview met Erick, leerkracht}
\label{ch:interviewErick}

%% TODO: Hoe ben je te werk gegaan? Verdeel je onderzoek in grote fasen, en
%% licht in elke fase toe welke stappen je gevolgd hebt. Verantwoord waarom je
%% op deze manier te werk gegaan bent. Je moet kunnen aantonen dat je de best
%% mogelijke manier toegepast hebt om een antwoord te vinden op de
%% onderzoeksvraag.
\section{Inleiding}
Erick Paez Montesinos werkte gedurende 9 jaar voor het Peruviaanse ministerie van onderwijs als educatieve medewerker. Daarnaast gaf hij les in verschillende onderwijsinstituten en werkte hij in een monitoraat dat andere leerkrachten op pedagogische vlak bijschoolt en begeleidt. Op dit moment geeft hij geen les meer, omdat hij een andere richting uitging, maar vorig schooljaar gaf hij les in educación primaria. Dat is het Peruviaanse basis onderwijs, waar kinderen tussen 6 en 11 jaar onder vallen \autocite{Nuffic2015}. Ook gaf hij les in educación inicial, educación secundaria en educación superior. Hij is iemand met veel ervaring binnen het onderwijs en kan ons relevante informatie bezorgen voor dit onderzoek. Dit interview werd, door de huidige maatregels tegen het Covid-19 virus, afgenomen vanop afstand.

 %Maestro Titular de Primaria
 In Figuur \ref{erick} is Erick te zien.
 
 \begin{figure}[h!]
 	\includegraphics[width=\textwidth]{../img/erick.jpeg}
 	\caption{Erick Paez Montesinos }
 	\label{erick}
 \end{figure}

\section{Interview}

\textbf{In welke school werkte je?}

\textit{Erick:} Ik werkte in een staatsschool in Apurímac. Dat is een regio net buiten Cusco. Daar werkte ik in instituut 50640, genaamd Sagrado corazón de Jesus.
%http://www.dreapurimac.gob.pe/inicio/images/ARCHIVOS_2019/CD-19/CD-2019-UGEL-COTABAMBAS.pdf

\textbf{Hoeveel leerlingen zitten er gemiddeld in de klassen waarin u lesgaf?}

\textit{Erick:} Er zitten normaal gezien 20 tot 25 leerlingen in elke klas. Op de hele school waren er 11 klassen, dus er waren ongeveer 220 tot 275 leerlingen op de school waar ik les gaf. %5:06

\textbf{Had je op jouw school computers ter beschikking? Welke apparatuur was er? Waren er problemen met deze apparatuur?}

\textit{Erick:} Er was één computerklas. Leerkrachten konden het lokaal boeken en het gebruik van computers in hun lessen verwerken. Deze computers werden specifiek gebruikt voor informaticalessen. Verder had de school ook laptops die door de overheid betaald werden. Deze laptops konden door klassen uitgeleend worden, en werden gebruikt als hulpmiddel tijdens andere lessen, zoals wiskunde of taal. De laptops hebben minder rekenkracht en zijn alleen voorzien van een aantal basis-functionaliteiten. Ze zijn gemaakt om educatieve programma's te gebruiken en informatie te raadplegen op het internet. Het zijn wit-groene laptops van het merk XO. De school had ongeveer 70 laptops van dit type. Op dit moment worden ze minder en minder gebruikt, omdat ze verouderd  zijn.

\textbf{Had jouw school een ICT leerkracht?}

\textit{Erick:} Ja, onze school had een ICT leerkracht. Hij was verantwoordelijk voor het controleren, en organiseren van alles wat met ICT te maken had. Normaal gezien worden deze leerkrachten geselecteerd en opgeleid door de educatieve koepel van het onderwijs district. Ik denk dat ongeveer 85\% van de publieke scholen in Peru een ICT leerkracht heeft. Bij ons op school was dit iemand die extra lessen had gevolgd, en zich had bijgeschoold tot leerkracht informatica. In andere scholen zijn er ook ICT leerkrachten die effectief informatica gestudeerd hebben.

\textbf{In hoeverre kunnen leerkrachten die geen ICT geven als vak om met informatica?}

\textit{Erick:} Er wordt verwacht van alle leerkrachten dat ze een basisniveau ICT hebben. Het komt echter vaak voor dat leerkrachten met veel onderwijservaring niet goed overweg kunnen met computers. Dat komt omdat ze door de jaren heen niet voldoende bijscholing kregen. Er wordt verwacht dat ze zichzelf bijscholen, maar dat gebeurt meestal niet.

\textbf{Als leerlingen informatica les krijgen, wat leren ze dan?}

\textit{Erick:} Ze leren niet echt specifiek applicaties te gebruiken maar vooral het basisgebruik van de computer: aan en uit zetten, iets op zoeken op het internet. Maar bijvoorbeeld Microsoft Excel of programmeren leren ze niet.

\textbf{Krijgen leerlingen huiswerk op de computer?}

\textit{Erick:} Ja, als kinderen huiswerk krijgen is dat meestal op papier, en gebruiken ze hun laptops als hulpmiddel. In kleine gemeenschappen, waar de mensen meestal arm zijn, en dus thuis geen computers hebben, zijn er scholen waar de leerlingen van educación primaria hun laptops mee naar huis mogen nemen. Daar kunnen ze via satelliet verbindingen, die gratis zijn, filmpjes kijken, informatie opzoeken en educatieve spelletjes spelen. 

\textbf{Worden de laptops dan niet gestolen? Wordt er geen misbruik gemaakt van het systeem?}

\textit{Erick:} Neen, mocht dit toch het geval zijn dan zal de school altijd uitvoerig onderzoeken wat er gebeurde. Dit gaat om educatieve laptops, dus echt interesse is er niet naar. Een leerling kan zijn laptop niet verkopen, want niemand zal ze kopen. 

\textbf{Zijn er problemen met de laptops die gebruikt worden?}

\textit{Erick:} Ja, eigenlijk wel. Op de school waar ik les gaf begonnen de laptops te verouderen, daarom werden ze minder en minder gebruikt. Hun schermen waren ook niet van de beste kwaliteit en ze hadden niet genoeg geheugen capaciteit. Ook zijn de scharnieren niet sterk genoeg zodat deze braken.

\textbf{Hoe komt het dat Peru niet verder staat op vlak van informatica, wat liep er fout in het verleden?}

\textit{Erick:} Het probleem is volgens mij groter dan informatica alleen. Vaak is de toegang tot internet beperkt in kleine gemeenschappen. Deze mensen kunnen niet op internet, en hebben ook geen financiële middelen om een computer aan te kopen. Volgens mij liggen economische problemen aan de basis van het informatica probleem. Er is geen geld voor computers. Niet bij de overheid, maar vaak ook niet bij de gezinnen zelf. 

Ik denk dat NGO's kunnen helpen om deze problematiek de wereld uit te helpen en om elke school te voorzien van informatica. Ik denk dat hier in Peru er een aantal goede NGO's zijn die ons een duw in de rug gaven, en dat dit ook zou kunnen op vlak van informatica. De overheid doet zijn best om ons vooruit te helpen, maar meestal is er onvoldoende budget omdat informatica niet bovenaan de agenda staat. Ik denk dat, door de recente uitbraak van het covid-19 virus, de ogen van de regering open gingen. Dat blijk uit het feit dat de overheid zeer recent veel computers aankocht om uit te delen aan kinderen, die dan op die manier vanop afstand les kunnen volgen. Ik denk dat dit misschien veel zal verbeteren aan de ondermaatse informaticakennis in het onderwijs. De staatsscholen kunnen niet zelf instaan voor de aankoop van hun computers, omdat dat dat voor hen gewoon te duur is, en omdat ze hier niet genoeg budget voor krijgen van de overheid. Nogmaals, ik denk dat de steun van NGO's veel zou kunnen betekenen.

\subsection{Conclusie}
Erick vertelt dat zijn school ook OLPC laptops heeft. De laptops hebben de mogelijkheid om via satelliet te verbinden met het internet. Het OLPC project is intussen gestopt, en de computers beginnen verouderd te raken. Recent nog kocht de overheid 840.000 laptops aan voor het onderwijs \autocite{Riofrio2020}.

Erick haalt twee belangrijke onderwerpen aan: het tekort aan bijscholing in onderwijs en de positieve invloed van NGO's. Hiermee zal zeker rekening gehouden worden in het beantwoorden van de onderzoeksvragen.