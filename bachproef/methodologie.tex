%%=============================================================================
%% Methodologie
%%=============================================================================

\chapter{Methodologie}
\label{ch:methodologie}

%% TODO: Hoe ben je te werk gegaan? Verdeel je onderzoek in grote fasen, en
%% licht in elke fase toe welke stappen je gevolgd hebt. Verantwoord waarom je
%% op deze manier te werk gegaan bent. Je moet kunnen aantonen dat je de best
%% mogelijke manier toegepast hebt om een antwoord te vinden op de
%% onderzoeksvraag.

Om een goed beeld te krijgen van hoe het onderwijs in Peru in elkaar zit en hoe informatica er bij aan te pas komt werd ervoor gekozen om verschillende specialisten te interviewen.  De specialisten hebben door studies en ervaring veel relevante kennis, elk binnen hun eigen domein. Ze werden gekozen om vanuit verschillende invalshoeken een kijk te krijgen op de probleemstelling. Er werden gelijkaardige vragen gesteld om de verschillen en gelijkenissen tussen de ervaringsdeskundigen bloot te leggen en duidelijk weer te kunnen geven.

De vragen bestaan uit 4 grote onderwerpen: 

\begin{itemize}
	\item De informatica infrastructuur op scholen
	\item Toepassen van informatica in de lessen
	\item De informatica leerkracht op school
	\item Hoe gaat Peru om met informatica
\end{itemize}

Na de interviews worden de uitspraken, gedaan door de geïnterviewde, gecontroleerd en gestaafd of weerlegt met reeds vergaarde bronnen uit de literatuurstudie en nieuwe informatie. Hierna volgt een conclusie met suggesties tot optimalisatie gebaseerd op de informatie verkregen uit de interviews.


