%%=============================================================================
%% Methodologie
%%=============================================================================

\chapter{Methodologie}
\label{ch:methodologie}

%% TODO: Hoe ben je te werk gegaan? Verdeel je onderzoek in grote fasen, en
%% licht in elke fase toe welke stappen je gevolgd hebt. Verantwoord waarom je
%% op deze manier te werk gegaan bent. Je moet kunnen aantonen dat je de best
%% mogelijke manier toegepast hebt om een antwoord te vinden op de
%% onderzoeksvraag.

Om een goed beeld te krijgen van hoe het onderwijs in Peru in elkaar zit en hoe informatica er bij aan te pas komt werd ervoor gekozen om verschillende specialisten te interviewen. Deze specialisten werden gekozen om vanuit verschillende invalshoeken een kijk te krijgen op de probleemstelling. Er werden gelijkaardige vragen gesteld om de verschillen en gelijkenissen tussen de ervaringsdeskundigen weer te geven.

De vragen bestaan uit 3 onderwerpen: 

\begin{itemize}
	\item De informatica infrastructuur
	\item Toepassen van informatica in de lessen
	\item De informatica leerkracht
	\item Hoe gaat Peru om met informatica
\end{itemize}

Na de interviews worden de uitspraken, gedaan door de geïnterviewde, gecontroleerd. De uitspraken worden gestaafd of weerlegt met reeds vergaarde bronnen uit de literatuurstudie en nieuwe informatie. Hierna volgt een conclusie met suggesties tot optimalisatie gebaseerd op de informatie verkregen uit de interviews.


