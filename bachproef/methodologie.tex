%%=============================================================================
%% Methodologie
%%=============================================================================

\chapter{Methodologie}
\label{ch:methodologie}

%% TODO: Hoe ben je te werk gegaan? Verdeel je onderzoek in grote fasen, en
%% licht in elke fase toe welke stappen je gevolgd hebt. Verantwoord waarom je
%% op deze manier te werk gegaan bent. Je moet kunnen aantonen dat je de best
%% mogelijke manier toegepast hebt om een antwoord te vinden op de
%% onderzoeksvraag.

Om een duidelijk beeld te krijgen van de stand van zaken van het onderwijs in Peru en hoe informatica er zijn weg vindt in het onderwijs werd ervoor gekozen om verschillende specialisten te interviewen. De specialisten hebben door studies en ervaring veel relevante kennis, elk binnen hun eigen domein. Ze werden gekozen om vanuit verschillende invalshoeken een zicht te krijgen op de problematiek. Er werden gelijkaardige vragen gesteld om de verschillen en gelijkenissen tussen de ervaringsdeskundigen bloot te leggen en duidelijk weer te geven.

De vragen die tijdens de verschillende interviews werden gesteld, zijn onder te verdelen in 4 grote onderwerpen: 

\begin{itemize}
	\item De informatica-infrastructuur op scholen
	\item Het toepassen van informatica in de lessen
	\item De informatica leerkracht op school
	\item Hoe gaat Peru om met informatica in het algemeen?
\end{itemize}

Na de interviews worden de uitspraken, gedaan door de geïnterviewde, gecontroleerd en gestaafd of weerlegd met reeds vergaarde bronnen uit de literatuurstudie en nieuwe bekomen informatie. Hierna volgt een conclusie met suggesties tot optimalisatie.

