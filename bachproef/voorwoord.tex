%%=============================================================================
%% Voorwoord
%%=============================================================================

\chapter*{Woord vooraf}
\label{ch:voorwoord}

%% TODO:
%% Het voorwoord is het enige deel van de bachelorproef waar je vanuit je
%% eigen standpunt (``ik-vorm'') mag schrijven. Je kan hier bv. motiveren
%% waarom jij het onderwerp wil bespreken.
%% Vergeet ook niet te bedanken wie je geholpen/gesteund/... heeft

Op 2 mei begon mijn stage in Cusco, Peru. Ik liep stage bij Añañau, een organisatie die zich inzet om kansarme kinderen en jongeren uit een lokale Quechua-gemeenschap een betere toekomst te geven. Añañau is voornamelijk een onderwijs project. Het project werd opgericht door Ellen Bosch en Sadith Paez Montesinos. Zij staan vandaag in voor de dagdagelijkse werking (het administratieve, educatieve, financiele luik). Ze sturen een 10 tal medewerkers en vrijwilligers aan. Ik koos voor deze stage omdat ik denk dat ik zelf veel kan bijleren over hoe het leven in een Zuid-Amerikaans land verloopt. Ook werken met de kinderen sprak me echt aan, in combinatie met mijn studie en passie: informatica. Als stagair informatica student bij Añañau, kreeg ik veel algemene IT vragen waarvoor ik een oplossing zocht. Daarnaast schreef ik er twee applicaties voor het project: een bibliotheek-applicatie en een applicatie om nieuwe stagiairs of vrijwilligers de kans te geven om zich te registreren. 

Ik wou graag mijn bachelorproef onderzoek doen rond iets waarmee ik in aanraking kwam tijdens mijn stage. Ik wist van een eerdere uitwisseling, na mijn middelbaar onderwijs in Ecuador, hoever het onderwijs daar stond en hoe ze daar omgingen met informatica. Ik wist dus min of meer waaraan ik me kon verwachten. Het was ideaal om het informatica-aspect uit mijn studies te combineren met het onderwijs-aspect van het project. 

Tijdens de stage die ik liep in Peru verspreidde het covid-19 virus zich over de wereld. Ook Peru werd getroffen. Ondanks het feit dat de meeste besmettingen in Lima te situeren waren en Cusco relatief gespaard bleef, was het beter om terug te keren naar België en mijn stage vanop afstand verder te zetten. Ook dit bachelorproef-onderzoek moest volledig vanop afstand gebeuren door de opgelegde maatregelen, wat uiteraard erg jammer was.

Tijdens het schrijven werd ik door een aantal mensen geholpen. Eerst en vooral wens ik mijn co-promotor Mevr. Ellen Bosch hartelijk te bedanken. Zij gaf me - voor ik startte met mijn onderzoek - goede raad en een duidelijke richting mee, waardoor ik veel beter wist wat me te doen stond. Op tussentijdse basis evalueerde ze samen met Dhr. Joeri Van Herreweghe, mijn promoter, wat ik had geschreven, wat ik heel erg apprecieerde. Bedankt! Daarnaast wens ik Mvr. Sadith Paez Montesinos te bedanken. Ze hielp me aan contactpersonen van wie een interview kon afnemen en bezorgde me informatie over mijn onderzoeksdomein. Ook een speciale bedanking aan Dhr. Erick Paez Montesinos en Mvr. Carmen Rosa Diaz Fonseca die ik kon interviewen tijdens het onderzoek, en aan mijn ouders om mij door dik en dun te steunen.

Ik wens u veel leesplezier toe,

Lucas Vermeulen

29 mei 2020, Gent