%%=============================================================================
%% Conclusie
%%=============================================================================

\chapter{Conclusie}
\label{ch:conclusie}

% onderzoeksvra(a)g(en). Wat was jouw bijdrage aan het onderzoeksdomein en
% hoe biedt dit meerwaarde aan het vakgebied/doelgroep? 
% Reflecteer kritisch over het resultaat. In Engelse teksten wordt deze sectie
% ``Discussion'' genoemd. Had je deze uitkomst verwacht? Zijn er zaken die nog
% niet duidelijk zijn?
% Heeft het onderzoek geleid tot nieuwe vragen die uitnodigen tot verder 
%onderzoek?

\section{Inleiding}
Uit de interviews kwamen een aantal duidelijke gelijkenissen en verschillen naar boven. Via deze informatie zal er in dit hoofdstuk een antwoord geformuleerd worden op volgende onderzoeksvragen:

$Onderzoeksvraag 1$: Wat is de huidige stand van zaken op vlak van ICT binnen het onderwijs in Peru, en hoe is het land hiertoe gekomen?

$Onderzoeksvraag 2$: Hoe kunnen deze knelpunten weggewerkt worden in de toekomst, en hoe is het land hiertoe gekomen?

\section{Wat is de huidige stand van zaken op vlak van ICT binnen het onderwijs in Peru?}
Uit de interviews kwamen een aantal duidelijke knelpunten naar boven deze worden nog kort uitgelegd.

% verschillen tussen de interviews bespreken, en afleiden naar grote varieteit


\subsection{Slechte basis fundering}
Uit de interviews bleek meerdere malen dat de basis fundering in veel gebieden in Peru niet in orde is. Dat kan komen door de typische geografie van het land maar ook door het wanbeleid van de regering. Als de men vooruit wil gaan op vlak van informatica, en vele andere vlakken zou de overheid  er voor moeten zorgen dat de basis fundering in orde is. Stomend water, rioleringen en Elektriciteit zouden echt en evidentie moeten zijn.

\subsection{Te weinig werkmiddelen vanuit het ministerie van onderwijs}
Voor de voorbije regeringen was onderwijs duidelijk geen prioriteit. Het is belangrijk dat ze begrijpen wat onderwijs kan betekenen voor de toekomst van een land. Doordat onderwijs geen prioriteit was, is de infrastructuur vaak niet goed. 

\subsection{Huidige informatica infrastructuur verouderd}
De Peruviaanse regering sprong op de kar toen de XO laptops geproduceerd werden. Dit was een goede zaak voor het land, omdat er veel leerlingen voor het eerst in aanraking kwamen met informatica. Ondanks de kwalen functioneerden de XO laptops goed toen ze werden aangekocht. Op dit moment gaan de zijn de laptops verouderd. Doordat ze zo ver achter de huidige norm staan zijn ze onbruikbaar geworden. Ze dienen niet meer waarvoor ze ooit voor bedoeld waren.

\subsection{Te kort aan informatica bijlessen voor leerkrachten}
Tijdens de interviews bleek dat leerkrachten goed worden opgevolgd in Peru. Ze krijgen bijles en worden regelmatig gecoacht, wat een erg goede zaak is. Zo zijn ze goed op de hoogte van hun eigen materie en kunnen ze kun kennis op een goede manier doorgeven aan hun leerlingen. Echter kregen ze nooit informaticabijllessen, wat erg jammer is. Jonge leerkrachten hebben vaak meer voeling met ICT dan leerkrachten met veel ervaring. Net voor hen zou het erg zinvol kunnen zijn om met deze nieuwe technologieën mee te zijn. Ook bleek er dat er soms geen informatica leerkracht op scholen zijn. Daarom zou het goed om het niveau van informatica kennis bij elke leerkracht op te krikken, zodat ze met hun kennis minstens een waardige vervangles informatica kunnen bieden aan de leerlingen. Als leerkrachten toch bijles informatica willen volgen, moeten ze dat op eigen houtje doen en er soms zelfs zelf voor betalen.

\subsection{Te kort aan informatica leerkrachten}
Niet elke school in Peru heeft een leerkracht informatica. Dat komt omdat er in een aantal regio's geen opleiding is tot informatica leerkracht. Vaak moeten er dus andere, onopgeleide leerkrachten inspringen of moet er gekeken worden naar ICT-specialisten om informatica les te geven. In het geval er gekozen wordt voor andere leerkrachten hebben die meestal geen bijles informatica gevolgd. Als er naar een specialist gekeken wordt zijn die vaak te duur voor de scholen. 

\subsection{Te veel armoede in het algemeen}
Er is te veel armoede in sommige regio's in Peru. Gezinnen hebben soms niet genoeg centen om rond te komen. Als dit het geval is zijn ze al zeker niet in staat om zelf een computer aan te kopen voor het gezin. Daardoor kunnen de kinderen thuis hun huiswerk niet maken op de computer onder toezicht en gaan ze vaak naar internetcafés. Daar is er geen toezicht en kan de aandacht sneller verdwijnen in andere zaken zoals videospelletjes of kunnen ze, in het slechtste geval, op kwaadaardige websites terecht komen.

\section{Hoe kunnen deze knelpunten weggewerkt worden in de toekomst?}

\subsection{Peruviaanse overheid}
%basisfundering, extra steun opleidingen aanbieden, leerkrachten bijles geven
De overheid zou naar de toekomst toe echt moeten investeren in onderwijs. Ze zouden minsten de basis fundering voor stromend water, rioleringen en elektriciteit moeten kunnen voorzien voor scholen, zodat projecten die vanuit de school worden georganiseerd kunnen doorgaan. Er zou ook meer budget moeten kunnen voorzien worden voor het ministerie van onderwijs. Als dat kan zou een nieuwe grootschalige aankoop van computers voor in scholen, zoals ze dat deden voor de XO-laptops, heel erg nuttig zijn voor de leerlingen. 

\subsection{Ministerie van onderwijs}
Het ministerie van onderwijs zou ervoor moeten zorgen dat men overal in Peru, een opleiding tot leerkracht ICT kan volgen. Daarnaast zou elke school moeten voorzien van een informatica leerkracht die de infrastructuur kan beheren en les kan geven aan de leerlingen. Ook het basis informatica niveau van andere leerkrachten zou omhoog gekrikt moeten worden. Dat zou perfect gaan via bijlessen op de manier zoals het nu al gebeurt. Indien er meer budget kan voorzien worden vanuit de 

\subsection{Steun via NGO's}
Steun via NGO's kan ook echt een verschil maken voor informatica in onderwijs. Zo denk ik enerzijds dat grote bedrijven kunnen helpen door kortingen te geven aan scholen en andere organisaties zonder winstoogmerk. Zo heeft Microsoft bijvoorbeeld op dit moment een programma, genaamd "Microsoft for Education", waarmee ze informatica in het onderwijs willen stimuleren. Dit soort initiatieven kunnen de hoge drempel, die zich meestal uit in het prijskaartje, weg halen voor scholen. 

Daarnaast zouden kleinschaligere initiatieven ook echt hun nut kunnen hebben. Zo zouden er bijvoorbeeld verouderde computers kunnen ingezameld worden. Meestal zijn verouderde computers te oud, omdat de software meer capaciteit nodig heeft. Daarom zou er een licht besturingssysteem kunnen ontworpen worden, speciaal gericht op kinderen en educatie. Die software kan dan geïnstalleerd worden op de oude computers, die dan verdeeld zouden kunnen  worden onder scholen. Zo zouden een hele hoop oude computers gerecycleerd kunnen worden, en in hun tweede leven hun nut kunnen bewijzen op scholen.

\section{Toekomstig onderzoek binnen dit domein}

Een aantal toekomstige onderzoeken binnen dit domein zouden kunnen zijn: 
\begin{itemize}
	\item Wat was de impact van het Covid-19 virus op informatica in het onderwijs in Peru?
	\item In hoeverre verbeteren schoolresultaten als er gebruik gemaakt wordt van informatica tijdens het leer proces?
\end{itemize}

