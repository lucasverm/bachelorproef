

\chapter{Conclusie}
\label{ch:conclusie}

% onderzoeksvra(a)g(en). Wat was jouw bijdrage aan het onderzoeksdomein en
% hoe biedt dit meerwaarde aan het vakgebied/doelgroep? 
% Reflecteer kritisch over het resultaat. In Engelse teksten wordt deze sectie
% ``Discussion'' genoemd. Had je deze uitkomst verwacht? Zijn er zaken die nog
% niet duidelijk zijn?
% Heeft het onderzoek geleid tot nieuwe vragen die uitnodigen tot verder 
%onderzoek?

\section{Inleiding}
Uit de interviews kwamen een aantal duidelijk gelijklopende, maar ook een aantal verschillende meningen naar voor. De drie interviews werden afgenomen van drie personen uit drie verschillende regio's. Hun ervaring, achtergrond, leeftijd, omgang van de scholen of projecten waren uiteraard verschillend, maar geven toch een vrij uniform beeld weer van de situatie in Peru. Via deze informatie zal er in dit hoofdstuk een antwoord geformuleerd worden op volgende onderzoeksvragen:

$Onderzoeksvraag$ 1: Wat is de huidige stand van zaken op vlak van ICT binnen het onderwijs in Peru, en hoe is het land hiertoe gekomen?

$Onderzoeksvraag$ 2: Hoe kunnen deze knelpunten weggewerkt worden in de toekomst, en hoe is het land hiertoe gekomen?

\section{Wat is de huidige stand van zaken op vlak van ICT binnen het onderwijs in Peru?}
Uit de interviews kwamen een aantal duidelijke knelpunten naar boven:

% verschillen tussen de interviews bespreken, en afleiden naar grote varieteit


\subsection{Slechte basis fundamenten}
Uit de interviews bleek meerdere malen dat de basis fundamenten (water, elektriciteit,... ) in veel gebieden in Peru niet beschikbaar is. Dat kan komen door de typische geografie van het land maar ook door het wanbeleid van de regering. Als men vooruit wil gaan op het vlak van informatica zou de overheid er voor moeten zorgen dat minstens de basisbehoeften in orde zijn. Stromend water, rioleringen en elektriciteit zouden overal een evidentie moeten zijn. Echter is dat vandaag niet overal het geval.

\subsection{Te weinig werkmiddelen vanuit het ministerie van onderwijs}
Voor de voorbije regeringen was onderwijs duidelijk geen prioriteit. Het is belangrijk dat een overheid begrijpt wat onderwijs kan betekenen voor de toekomst van een land. Doordat onderwijs geen prioriteit was werden er te weinig werkmiddelen en budgetten vrijgemaakt.

\subsection{Huidige informatica infrastructuur verouderd}
De Peruviaanse regering sprong in 2007 op de kar toen de XO laptops geproduceerd en verdeeld werden. Dit was een goede zaak voor het land, omdat er veel leerlingen voor het eerst in aanraking kwamen met informatica. Ondanks de kwalen functioneerden de XO laptops goed toen ze werden aangekocht. Op dit moment zijn de laptops echter compleet verouderd. Ze zijn daardoor compleet onbruikbaar geworden.

\subsection{Te kort aan informatica-bijscholing voor leerkrachten}
Vaak zijn er geen opgeleide informatica leerkrachten op scholen. Daarom zou het goed zijn om het niveau van informaticakennis bij elke leerkracht op te krikken, zodat ze met hun kennis minstens een volwaardige les informatica kunnen 	aanbieden aan de leerlingen. Als leerkrachten toch bijscholing informatica willen volgen, moeten ze dat op eigen houtje doen en er soms zelfs zélf voor betalen. Jonge leerkrachten hebben vaak meer voeling met ICT dan leerkrachten met een lange staat van dienst. Net voor hen zou het erg zinvol kunnen zijn om met deze nieuwe technologieën mee te zijn. 

\subsection{Te kort aan informatica leerkrachten}
Niet elke school in Peru heeft een leerkracht informatica. Dat komt omdat er in een aantal regio's geen opleiding is tot informatica leerkracht. Vaak moeten er dus andere, onopgeleide leerkrachten inspringen of moet er gekeken worden naar ICT-specialisten om informatica les te geven. In het geval er gekozen wordt voor andere leerkrachten hebben die meestal geen bijscholing informatica gevolgd. Als er naar een specialist gekeken wordt zijn die vaak te duur voor de scholen. 

\subsection{Te veel armoede in het algemeen}
Er is te veel armoede in sommige regio's in Peru. Gezinnen hebben daar niet genoeg financiële middelen om rond te komen. Als dit het geval is zijn ze al zeker niet in staat om een computer aan te kopen voor het gezin. Daardoor kunnen de kinderen thuis hun huiswerk niet maken en gaan ze vaak naar internetcafés. Daar is er geen toezicht en kan de aandacht sneller afleiden naar andere zaken zoals videospelletjes of kunnen ze, in het slechtste geval, op kwaadaardige websites terecht komen.

\section{Hoe kunnen deze knelpunten weggewerkt worden in de toekomst?}

\subsection{Peruviaanse overheid}
%basisfundering, extra steun opleidingen aanbieden, leerkrachten bijles geven
De overheid zou in de toekomst werk moeten maken van investeren in het onderwijs. Ze zouden minsten de basis fundamenten voor stromend water, rioleringen en elektriciteit moeten voorzien voor scholen, zodat projecten die vanuit de school worden georganiseerd kunnen doorgaan. Er zou ook meer budget moeten voorzien worden voor het ministerie van onderwijs. Als dat kan zou een nieuwe grootschalige aankoop van computers voor in scholen, zoals ze dat deden voor de XO-laptops, heel erg nuttig zijn voor de leerlingen. 

\subsection{Ministerie van onderwijs}
Het ministerie van onderwijs zou ervoor moeten zorgen dat men overal in Peru een opleiding tot leerkracht ICT kan volgen. Daarnaast zou elke school moeten voorzien in een informatica leerkracht die de infrastructuur kan beheren en les kan geven aan de leerlingen. Ook het basis informatica niveau van andere leerkrachten zou omhoog gekrikt moeten worden. Dat zou perfect gaan via bijscholing op de manier zoals het nu al gebeurt. Indien er meer budget kan voorzien worden vanuit de overheid zouden deze doelstellingen kunnen behaald worden.

\subsection{Steun via NGO's}
Steun via NGO's kan een verschil maken voor informatica in onderwijs. Grote bedrijven kunnen helpen door kortingen te geven aan scholen en andere organisaties zonder winstoogmerk. Zo heeft Microsoft bijvoorbeeld op dit moment een programma, genaamd "Microsoft for Education", waarmee ze informatica in het onderwijs willen stimuleren. Door het verlagen van het prijskaartje kunnen dit soort initiatieven de aankoop van IT materiaal faciliteren.

%Dit soort initiatieven kunnen de vaak te hoge drempel, die zich meestal uit in het veel te hoge prijskaartje, weg halen voor scholen. 

Daarnaast zouden kleinschaligere initiatieven ook hun nut kunnen hebben. Zo zouden er bijvoorbeeld verouderde computers kunnen ingezameld worden. Meestal zijn verouderde computers te oud, omdat de nieuwe software meer capaciteit nodig heeft. Daarom zou er een licht besturingssysteem kunnen ontworpen worden, speciaal gericht op kinderen en educatie. Die software kan dan geïnstalleerd worden op oude computers, die dan verdeeld zouden kunnen worden onder scholen. Zo zouden een hele hoop oude computers gerecycleerd kunnen worden, en toch nuttig zijn in het onderwijs.

\section{Toekomstig onderzoek binnen dit domein}

Een aantal toekomstige onderzoeken binnen dit domein zouden kunnen zijn: 
\begin{itemize}
	\item Wat was de impact van het Covid-19 virus op informatica in het onderwijs in Peru?
	\item In hoeverre verbeteren schoolresultaten als er gebruik gemaakt wordt van informatica tijdens het leerproces?
\end{itemize}

