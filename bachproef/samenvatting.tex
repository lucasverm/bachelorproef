% thesis) synthese van het document.
%
% Deze aspecten moeten zeker aan bod komen:
% - Context: waarom is dit werk belangrijk?
% - Nood: waarom moest dit onderzocht worden?
% - Taak: wat heb je precies gedaan?
% - Object: wat staat in dit document geschreven?
% - Resultaat: wat was het resultaat?
% - Conclusie: wat is/zijn de belangrijkste conclusie(s)?
% - Perspectief: blijven er nog vragen open die in de toekomst nog kunnen
%    onderzocht worden? Wat is een mogelijk vervolg voor jouw onderzoek?
%
% LET OP! Een samenvatting is GEEN voorwoord!

%%---------- Nederlandse samenvatting -----------------------------------------
%
% TODO: Als je je bachelorproef in het Engels schrijft, moet je eerst een
% Nederlandse samenvatting invoegen. Haal daarvoor onderstaande code uit
% commentaar.
% Wie zijn bachelorproef in het Nederlands schrijft, kan dit negeren, de inhoud
% wordt niet in het document ingevoegd.

\chapter*{Samenvatting}
\label{ch:samenvatting}
Peru investeerde in 2007 in het One Laptop Per Child (OLPC) project om informatica in het onderwijs te stimuleren. Dit is het grootste OLPC-programma dat tot nog toe plaats vond. Er zouden 902.000 educatieve computers aangekocht zijn. Dat wierp zijn vruchten af, maar is de impact van deze investering nog steeds zichtbaar in het onderwijs? Deze bachelorproef onderzoekt de huidige stand van zaken omtrent informatica in het onderwijs in Peru. Ook wordt er gezocht naar suggesties tot optimalisatie. 

Via interviews, afgenomen bij vakspecialisten, werd geconstateerd dat er zeker nog werk is op het vlak van het aanleren van informatica-skills binnen educatie in Peru. Er is te weinig basis fundament voorzien voor de scholen, de informatica infrastructuur is verouderd en het ministerie van onderwijs heeft te weinig middelen. Ook is er een tekort aan ICT leerkrachten en informatica opleidingen voor gewone leerkrachten. De algemene armoede van het ontwikkelingsland is ook een probleem dat meespeelt in de digitalisering van het onderwijs. 

Mogelijke oplossingen kunnen geboden worden door de Peruviaanse overheid en NGO's. De overheid zou meer geld kunnen uittrekken voor onderwijs om de infrastructuur te verbeteren, meer ICT leerkrachten aan te stellen en opleidingen informatica te voorzien voor de leerkrachten. Grote NGO's kunnen hulp bieden door hun goederen of diensten aan goedkopere tarieven aan te bieden of die gratis beschikbaar te stellen voor organisaties zonder winstoogmerk. Kleine NGO's kunnen ook het verschil maken door bijvoorbeeld verouderde bedrijfscomputers in te zamelen, deze te formatteren, en er een zelf ontwikkelde educatiebesturingssysteem op te installeren. Hierna kunnen de computers verdeeld worden onder de verschillende Peruviaanse scholen die er nood aan hebben. 

Dit onderzoek liep tijdens de uitbraak van het covid-19 virus. Deze uitzonderlijke situatie bracht met zich mee dat de Peruviaanse overheid middelen vrijmaakte met het doel kinderen tijdens de pandemie de mogelijk te bieden om van thuis uit les te volgen.  Het zou erg interessant zijn om - op een later tijdstip - te onderzoeken welke impact de pandemie op het probleem van informatica binnen onderwijs heeft gehad.


