\chapter{Inleiding}
\label{ch:inleiding}

Onderwijs is een belangrijke factor in de groei en ontwikkeling van een land. Elk onderwijssysteem van elk land heeft andere kenmerken, noden en kwaliteiten. Informatica is in de om het even welke maatschappij nergens meer weg te denken. Op scholen moet men alle trends kunnen volgen, wat niet altijd even eenvoudig blijkt. Dit onderzoek richt zich op informatica in educatie in Peru. Het bekijkt hoe het staat met informatica in het onderwijs, en er worden suggesties tot optimalisatie gedaan.

Peru is een Zuid-Amerikaans ontwikkelingsland. Er heerst zowel in de grote steden als in het binnenland veel armoede. Vooral in het staatsonderwijs zijn de gevolgen daarvan duidelijk. Het onderwijs in Peru bestaat uit publieke scholen en private scholen. Publieke scholen of staatsscholen zijn toegankelijk voor iedereen. Voor private scholen moeten kinderen toegelaten worden en meestal maandelijks betalen. In het algemeen geldt er dat de kinderen van arme gezinnen naar staatsscholen gaan en de kinderen van rijkere gezinnen naar private scholen. Omdat staatsscholen voor iedereen toegankelijk zijn is dit onderzoek volledig gebaseerd op het staatsonderwijs. Er zal onderzoek gedaan worden naar de huidige stand van zaken van ICT binnen het staatsonderwijs in Peru en er zal naar optimalisaties gezocht worden. Dit zal gebeuren op basis van interviews waarna er conclusies getrokken zullen worden.

%De inleiding moet de lezer net genoeg informatie verschaffen om het onderwerp te begrijpen en in te zien waarom de onderzoeksvraag de moeite waard is om te onderzoeken. In de inleiding ga je literatuurverwijzingen beperken, zodat de tekst vlot leesbaar blijft. Je kan de inleiding verder onderverdelen in secties als dit de tekst verduidelijkt. Zaken die aan bod kunnen komen in de inleiding~\autocite{Pollefliet2011}:

%\begin{itemize}
%  \item context, achtergrond
%  \item afbakenen van het onderwerp
%  \item verantwoording van het onderwerp, methodologie
%  \item probleemstelling
%  \item onderzoeksdoelstelling
%  \item onderzoeksvraag
%  \item \ldots
%\end{itemize}

\section{Probleemstelling}
\label{sec:probleemstelling}
Veel ontwikkelingslanden hebben te maken met een zogenaamde digital gap. Ze lopen achter op vlak van informatica tegenover beter ontwikkelde landen. Informatica is erg belangrijk in de ontwikkeling van een land. Onderwijs kan aan de basis liggen om dit probleem te verhelpen. Door kinderen op te leiden en ze te voorzien van know-how kan het land de digital gap proberen te dichten. Zowel de overheid als leerlingen en scholen hebben baat bij een goede implementatie van informatica in het onderwijs. 

%Uit je probleemstelling moet duidelijk zijn dat je onderzoek een meerwaarde heeft voor een concrete doelgroep. De doelgroep moet goed gedefinieerd en afgelijnd zijn. Doelgroepen als ``bedrijven,'' ``KMO's,'' systeembeheerders, enz.~zijn nog te vaag. Als je een lijstje kan maken van de personen/organisaties die een meerwaarde zullen vinden in deze bachelorproef (dit is eigenlijk je steekproefkader), dan is dat een indicatie dat de doelgroep goed gedefinieerd is. Dit kan een enkel bedrijf zijn of zelfs één persoon (je co-promotor/opdrachtgever).

\section{Onderzoeksvragen}
\label{sec:onderzoeksvraag}

%Wees zo concreet mogelijk bij het formuleren van je onderzoeksvraag. Een onderzoeksvraag is trouwens iets waar nog niemand op dit moment een antwoord heeft (voor zover je kan nagaan). Het opzoeken van bestaande informatie (bv. ``welke tools bestaan er voor deze toepassing?'') is dus geen onderzoeksvraag. Je kan de onderzoeksvraag verder specifiëren in deelvragen. Bv.~als je onderzoek gaat over performantiemetingen, dan 

$Onderzoeksvraag 1$: Wat is de huidige stand van zaken op vlak van ICT binnen het onderwijs in Peru, en hoe is het land hiertoe gekomen?

$Onderzoeksvraag 2$: Hoe kunnen deze knelpunten weggewerkt worden in de toekomst?

\section{Onderzoeksdoelstelling}
\label{sec:onderzoeksdoelstelling}

%Wat is het beoogde resultaat van je bachelorproef? Wat zijn de criteria voor succes? Beschrijf die zo concreet mogelijk. Gaat het bv. om een proof-of-concept, een prototype, een verslag met aanbevelingen, een vergelijkende studie, enz.

De doelstelling van dit onderzoek is het erkennen en correct formuleren van de pijnpunten van ICT binnen het onderwijs in Peru. Naar het einde toe worden er ook suggesties gedaan om de knelpunten te optimaliseren.

\section{Opzet van deze bachelorproef}
\label{sec:opzet-bachelorproef}

% Het is gebruikelijk aan het einde van de inleiding een overzicht te
% geven van de opbouw van de rest van de tekst. Deze sectie bevat al een aanzet
% die je kan aanvullen/aanpassen in functie van je eigen tekst.

Deze bachelorproef is als volgt opgebouwd:

In Hoofdstuk~\ref{ch:stand-van-zaken} wordt een overzicht gegeven van de stand van zaken binnen het onderzoeksdomein, op basis van een literatuurstudie.

In Hoofdstuk~\ref{ch:methodologie} wordt de methodologie toegelicht en worden de gebruikte onderzoekstechnieken besproken om een antwoord te kunnen formuleren op de onderzoeksvragen.

In Hoofdstuk~\ref{ch:interviewSadith} wordt een interview met Sadith Paez Montesinos, één van de mede-oprichtsters van Añañau besproken.

In Hoofdstuk~\ref{ch:interviewRosa} wordt een interview met Rosa Diaz Fonseca, een ex-medewerker van het Peruviaanse ministerie van onderwijs en schooldirectrice besproken.

In Hoofdstuk~\ref{ch:interviewErick} wordt een interview met Erick Paez Montesinos, een ex-medewerker van het Peruviaanse ministerie van onderwijs en leerkracht besproken.

In Hoofdstuk~\ref{ch:conclusie}, tenslotte, worden conclusies getrokken en wordt een antwoord geformuleerd op de verschillende onderzoeksvragen. Daarbij wordt ook een aanzet gegeven voor toekomstig onderzoek binnen dit domein.