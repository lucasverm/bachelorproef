%%=============================================================================
%% InterviewSadith
%%=============================================================================

\chapter{Interview met Sadith Van Añañau ORG}
\label{ch:interviewMetSadith}

%% TODO: Hoe ben je te werk gegaan? Verdeel je onderzoek in grote fasen, en
%% licht in elke fase toe welke stappen je gevolgd hebt. Verantwoord waarom je
%% op deze manier te werk gegaan bent. Je moet kunnen aantonen dat je de best
%% mogelijke manier toegepast hebt om een antwoord te vinden op de
%% onderzoeksvraag.
\section{Inleiding}
Om een antwoord te kunnen bieden op de eerst onderzoeksvraag, werd ervoor gekozen om een interview af te nemen met Sadith Paez Montesinos. Sadith is één van de oprichtsters van Añañau. Añañau is een non-profit en niet-gouvernamentele organisatie die in december 2014 werd opgericht in Cusco, Peru. De organisatie werkt met kinderen en jongeren tussen 4 en 18 jaar oud die in situaties van extreme armoede en onstabiele familiesituaties leven in het district San Jeronimo, een buitenwijk van Cusco. Sinds januari 2015 is de organisatie als officiële NGO erkend in Peru. Omdat ze geloven in de kracht van goed en inclusief onderwijs als springplank naar een betere toekomst, is Añañau in de eerste plaats een educatief project. Via huiswerkbegeleiding en spelend leren stimuleert Añañau de ontwikkeling van de kinderen en trachten ze nieuwe kansen te creëren. \autocite{Ananau2020}

Door haar ervaring in het werkveld beschikt Sadith over relevante info omtrent het onderwijs in Peru. 

\section{Interview}
\textbf{Hebben de meeste staatsscholen in Peru computers ter beschikking op school? Welke apparatuur is er? Zijn er gebreken aan deze apparatuur? Is de apparatuur bruikbaar?}

\textit{Sadith:} Niet alle scholen hebben computers. Van de scholen die zich in de binnenstad bevinden heeft volgens mij 80\% van de scholen computers ter beschikking. Bij de scholen buiten de stad, in het binnenland van Peru is dat volgens mij 40\% van de scholen. Als een school dan computers heeft, zijn dat er ongeveer 10 of 20 computers per 200 leerlingen.

\textbf{Hoe gebruiken de kinderen de computers om te leren?}

\textit{Sadith:} Normaal gezien gebruiker ze de computers om computer les te krijgen. De leerlingen gaan dus niet voor andere vakken aan de slag met informatica. 

\textbf{Wat leren de kinderen dan tijdens de computerlessen?}

\textit{Sadith:} Ze leren specifieke programma's gebruiken. Ze leren niet programmeren, maar leren vooral praktisch werken met bijvoorbeeld Word en Excel.

\textbf{Passen de staatsscholen informatica binnen hun curriculum nog op andere manieren toe, zoals met smartborden of tablets?}

\textit{Sadith:} Smartborden hebben de scholen niet. Ongeveer 5 jaar geleden werd een project opgestart waarbij tablets op scholen gestimuleerd werden. De tablets werkten in de grote steden, maar in het binnenland was er meestal geen verbinding. Hier in Peru moeten we elke 5 jaar gaan stemmen en verandert de regering. Het komt vaak voor dat nieuwe regeringen andere aanpakken hebben, en niet meer in deze projecten willen investeren. Na de verandering van de regering werd het tablet project stop gezet.

Nu sinds kort, zitten we in de huidige covid-19 crisis. Daardoor investeert de overheid op dit moment graag in ICT projecten voor het onderwijs, omdat alle leerlingen thuis moet blijven, en thuis moet studeren. Onlangs kochten ze volgens mij nog 13.000 computers aan die zich via satellieten kunnen verbinden met het internet. Deze computers zouden dienen om aan studenten uit te delen, zodat ze thuis zelf op de computer voor school kunnen werken. 

\textbf{Moeten kinderen die naar staatsscholen gaan soms huiswerk maken op computers? Wat moeten ze doen als ze geen computer hebben?}

\textit{Sadith:} Ja, de kinderen moeten thuis huiswerk maken op computers. Ze krijgen huiswerk op de computer van al hun vakken. Als ze thuis geen computer hebben gaan ze naar een internetcafé. Daar kunnen ze computers gebruiken voor 1 sol per uur (ongeveer 0,25 eurocent) en printen. Hier in Peru zijn er veel internetcafés. Een nadeel is wel dat de kinderen er vaak in aanraking komen met videospelletjes. Ze worden verleid, zijn vaak afgeleid, en komen niet aan huiswerk maken toe. 

\textbf{Zijn de leerkrachten voldoende opgeleid om computers te gebruiken?}

\textit{Sadith:} Niet alle leerkrachten kunnen met computers werken. Dit is een groot probleem hier. Er zijn leerkrachten die al veel ervaring hebben, en bijna op pensioen gaan. Deze generatie leerkrachten heeft veel minder voeling met de nieuwe technologieën.

\textbf{Zijn er dan geen bijscholingen voor die leerkrachten?}

\textit{Sadith:} Voor informatica moeten de leerkrachten zichzelf bijscholen. Dit wordt niet georganiseerd door de overheid. Er wordt dus verwacht dat de leerkrachten zelf op zoek gaan, en ervoor zorgen dat ze zelf op de hoogte zijn. In de relativiteit loop dit dus vaak fout en zijn er leerkrachten waarvan de computerkennis heel erg beperkt is. Wel bestaan er platformen die lessen geven. Die lessen zijn voor alle inwoners, en niet specifiek gericht op lesgevers.

\textbf{Zijn er ICT-leerkrachten, en zijn ze opgeleid?}

\textit{Sadith:} Hier in Cusco kan je volgens mij niet studeren voor ICT leerkracht. Vaak worden ICT leerkrachten aangesteld die ofwel gewoon veel weten van ICT, en hiervoor dus niet opgeleid zijn of er worden ICT specialisten gezocht die les willen geven, maar dat is heel erg duur en vaak onbetaalbaar.

\textbf{Is er programmatuur die kan gebruikt worden op de computers door de kinderen, zoals een taal- of rekenprogramma?}

\textit{Sadith:} Hier in Cusco kan je volgens mij niet studeren voor ICT leerkracht. Vaak worden ICT leerkrachten aangesteld die ofwel gewoon veel weten van ICT, en hiervoor dus niet opgeleid zijn of er worden ICT specialisten gezocht die les willen geven. Meestal willen de specialisten geen les geven, en kunnen ze meer verdienen door ergens anders aan de slag te gaan. Als een school er dan toch voor kiest om een specialist in dienst te nemen moeten ze veel betalen en is dit meestal onbetaalbaar.
	
\textbf{Zijn er digitale omgevingen waarop kinderen kunnen aanloggen?}

\textit{Sadith:} Vroeger niet, nu sinds een maand is de overheid begonnen met het ontwikkelen van een platform. Door de recente uitbraak van het covid-19 virus voorziet de overheid een platform waarop kinderen van thuis uit kunnen les volgen. Het platform heet aprendo en casa (https://aprendoencasa.pe/). Natuurlijk bereikt het platform alleen kinderen die toegang hebben tot een computer met internet. Verder worden er nu ook lessen uitgezonden op de radio, en televisie kanalen voor de overheid. Elke leeftijdsgroep heeft een tijdsspanne waarop ze dagelijks lessen kunnen volgen. Een specifiek platform voor elke school, waarop de leerlingen kunnen inloggen bestaat naar mijn weten niet. 

Op dit moment, door de pandemie, proberen leerkrachten vooral contact te houden met hun leerlingen via WhatsApp-groepen.

\textbf{Wat zijn de knelpunten op vlak van informatica in het onderwijs in Peru? Hoe kunnen deze knelpunten weggewerkt worden?}

\textit{Sadith:} Volgens mij zijn er 3 grote problemen: eerst en vooral was educatie geen prioriteit voor de regeringen in het verleden. Ze investeerden niet echt in scholen en educatie. Als er nieuwe scholen werden gebouwd, was er meestal geen budget meer vrij om computers aan te schaffen. Als de regeringen al investeert in educatie, zagen ze liever mooie gebouwen, dan computers op scholen. Volgens mij is geld dus de hoofdreden. Verder is vooral de infrastructuur een groot probleem. Vele scholen buiten de stad hebben geen internet aansluiting, door de moeilijke ligging, en de bergachtige geografie van Peru. Door alle bergen is het vaak heel gecompliceerd, en moeten particulieren zelf antennes aanschaffen om op internet te kunnen. Voor scholen zijn er dus veel kosten, en daar hebben ze meestal geen geld voor. Daardoor zijn computers onnuttig, omdat ze niet functioneel zijn. De huidige regering doet zijn best om ICT bij kinderen op scholen te ondersteunen. Zoals ik al zei kochten ze al grote aantallen computers voor aan de kinderen die geen toegang hebben tot informatica te geven.

Het tweede probleem is, zoals ik al zei, dat telkens als de regering wijzigt er een ander plan van aanpak is. Soms zijn er heel goede projecten die bij een regeringswissel gestopt worden. Zo ken een regering dus 5 jaar lang een project opzetten en kan de volgende regering te niet doen, wat erg jammer is. 

Het laatste probleem is dat er weinig informatica leerkrachten zijn, zoals ik eerder uitlegde. Er is hier in Cusco geen opleiding, en specialisten zijn erg duur.

Een oplossing zou kunnen zijn om, zoals wij doen op het project Añañau, krachten van mensen te bundelen en samen te zoeken naar budgetten waarmee er doelen kunnen bereikt worden. 

\textbf{Hoe komt het dat Peru niet verder staat op vlak van informatica, wat liep er fout in het verleden?}

\textit{Sadith:} Vaak is corruptie een probleem geweest in het verleden volgens mij. De huidige regering heeft de aankoop van computers voor op scholen duidelijk geprioriteerd, maar vroeger was dat niet het geval. Regeringen probeerden vaak geld in eigen zak te steken, en dat liep helemaal fout. 
 
\subsection{Besluit}
% combinatie met literatuur
% nog een interview?

Sadith haalde tijdens het gesprek aan dat de overheid recentelijk vele computers aankocht die verbinding maken met een sateliet om op verbinding met het internet te kunnen maken. Op 18 april verspreidde het ministerie van onderwijs een persbericht waarin ze vertelden dat ze 840.000 tablets aankochten met mobiel internet voor schoolkinderen in afgelegen landelijke en stedelijke gebieden. Dit, zodat ze de kinderen kunnen blijven studeren en de digitale kloof kan worden verkleind. Op deze manier kunnen de kinderen onderwijs vanop afstand volgen.\autocite{Minedu2020} Nadat ik Sadith confronteerde met het persbericht zei ze me dat ze het over dit bericht had in het interview. 

%aprendo en casa 
Ook vertelt Sadith over 'aprendo en casa' (vertaald: ik leer thuis). Dat is een programma dat de regering aanbied om de schoolgaande kinderen te voorzien van thuis onderwijs. Het \autocite{Minedu2020a}
%Studeren in cusco voor ict leerkracht ?
