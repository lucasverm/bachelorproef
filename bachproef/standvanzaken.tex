\chapter{\IfLanguageName{dutch}{Stand van zaken}{State of the art}}
\label{ch:stand-van-zaken}

% Tip: Begin elk hoofdstuk met een paragraaf inleiding die beschrijft hoe
% dit hoofdstuk past binnen het geheel van de bachelorproef. Geef in het
% bijzonder aan wat de link is met het vorige en volgende hoofdstuk.

% Pas na deze inleidende paragraaf komt de eerste sectiehoofding.

%Dit hoofdstuk bevat je literatuurstudie. De inhoud gaat verder op de inleiding, maar zal het onderwerp van de bachelorproef *diepgaand* uitspitten. De bedoeling is dat de lezer na lezing van dit hoofdstuk helemaal op de hoogte is van de huidige stand van zaken (state-of-the-art) in het onderzoeksdomein. Iemand die niet vertrouwd is met het onderwerp, weet nu voldoende om de rest van het verhaal te kunnen volgen, zonder dat die er nog andere informatie moet over opzoeken \autocite{Pollefliet2011}.

%Je verwijst bij elke bewering die je doet, vakterm die je introduceert, enz. naar je bronnen. In \LaTeX{} kan dat met het commando \texttt{$\backslash${textcite\{\}}} of \texttt{$\backslash${autocite\{\}}}. Als argument van het commando geef je de ``sleutel'' van een ``record'' in een bibliografische databank in het Bib\LaTeX{}-formaat (een tekstbestand). Als je expliciet naar de auteur verwijst in de zin, gebruik je \texttt{$\backslash${}textcite\{\}}.
%Soms wil je de auteur niet expliciet vernoemen, dan gebruik je \texttt{$\backslash${}autocite\{\}}. In de volgende paragraaf een voorbeeld van elk.

%\textcite{Knuth1998} schreef een van de standaardwerken over sorteer- en zoekalgoritmen. Experten zijn het erover eens dat cloud computing een interessante opportuniteit vormen, zowel voor gebruikers als voor dienstverleners op vlak van informatietechnologie~\autocite{Creeger2009}.

%-----------------------------------
% BESPREKING ELLEN

%waar staat peru in ontwikkeling op vlak van ict
%digital gap
%duurzame ontwikkelingsdoelen agenda 2030 vn = sdg 
%17 logo's
%SDG --> ICT
%milenium doelstellingen 2015
%
%peru in het algemeen
%als ontwikkelingsland --> vn/sdg
%hegt onderwijs in peru
%ict 
%digital gab 
%projecten in het verleden/organisaties ict hulp?
%
%
%vlirous - project beurs - peru project -computers recycleren

\section{Peru: een bloemlezing}
\subsection{Geografisch}
Peru ligt aan de westkust van zuid-Amerika. Het grenst in het noorden aan Colombia en Ecuador, in het oosten aan Bolivia en Brazilië, en in het zuiden aan Chili. Aan de westkust van het land ligt de Grote Oceaan. Lima is de hoofdstad van het land, en telt ruim 10 miljoen inwoners. Qua oppervlakte is Peru 42 keer groter dan België, het meet ongeveer 1.285.216 km\textsuperscript{2}. 

\subsubsection{Landschap}
Peru bestaat uit 3 verschillende landschapssoorten: het kustgebied, de bergen en de jungle. De kust van Peru bestaat uit woestijn die tussen de zee en de bergen ligt. De bergen bestaan vooral uit het Andes gebergte, en daarachter ligt de jungle. Die kan opgesplitst worden in 2 verschillende stukken. Als eerste is er het hoge regenwoud. Dat ligt op 700 meter hoogte. Ten tweede is er de amazone jungle, waar de Amazone rivier door stroomt. \autocite{ToPeru2020}

\subsection{Demografisch}
Peru heeft op dit moment 32 miljoen inwoners \autocite{Overheid2020}. Officieel worden er 3 talen gesproken: Spaans, Quechua en Aymara. Aymara is een taal die tot de familie van Aru behoort. Ze wordt gesproken door ongeveer 450.000 mensen. \autocite{CulturaPeru2020} Ze wordt ook gesproken in Bolivia en Noord-Argentinië en Chili. In de Amazone worden bijna 70 verschillende onofficiele, lokale talen gesproken. \autocite{dosmanosperu}

\subsection{Cultuur}
De nationale feestdag van Peru valt op 28 Juli. In Peru worden deze dagen de Fiestas Patrias genoemd. \autocite{dosmanosperu2018} 

\subsection{Economie}
In Peru betaald men met de Peruviaanse sol. Op dit moment is 1 euro ongeveer gelijk aan 3,82 Peruviaanse sol.

\section{Ontwikkelingshulp}

\subsection{Verenigde Naties}
De Verenigde Naties is een internationale organisatie die in 1945 is opgericht. De organisatie bestaat momenteel uit 193 lidstaten. Zowel België als Peru werden in 1945 lid van de verenigde naties. De verenigde naties heeft een handvest dat werd ondertekend op 26 juni 1945, in San Francisco. Het handvest trad in werking op 24 oktober 1945.Vanwege de bevoegdheden in het handvest en het unieke internationale karakter, kunnen de Verenigde Naties actie ondernemen tegen de problemen waarmee de mensheid in de 21e eeuw wordt geconfronteerd, zoals vrede en veiligheid, klimaatverandering, duurzame ontwikkeling, mensenrechten, ontwapening, terrorisme, humanitaire hulp en noodsituaties op gezondheidsgebied, gendergelijkheid, bestuur, voedselproductie en meer. \autocite{Nations2020}

Op 1 januari 2017 volgde de Portugese socialistische politicus António Guterres, Ban Ki-moon op als Secretatis-generaal van de organisatie. 

De belangrijkste organen van de VN zijn de Algemene Vergadering, de Veiligheidsraad, de Economische en Sociale Raad, de trustschapsraad, het Internationaal Gerechtshof en het VN-secretariaat. Ze werden allemaal opgericht in 1945 toen de VN werd opgericht. 

De vn bestaat uit vele programma's, fondsen en gespecialiseerde organisaties, allemaal met hun eigen leiderschap en budget. Een aantal bekende zijn Unicef, het Internationaal monetair fonds (IMF), Unesco, de wereld gezondheid (WHO). Natuurlijk bestaan er nog anderen. In dit onderzoek zal vooral het verenigde naties ontwikkelingsprogramma (UNDP) aan bod komen. Zoals de naam doet vermoeden houd dit deel van de verenigde naties zich bezig met ontwikkelingshulp.

\subsubsection{UNDP: Het Verenigde Naties ontwikkelingsprogramma}
Zoals eerder vermeld doet de verenigde Naties mee aan ontwikkelingshulp. Dit doen ze via een apart programma, het United Nations Development Programme (UNDP). Het ontwikkelingsprogramma van de Verenigde Naties is het wereldwijde ontwikkelingsnetwerk van de VN en pleit voor verandering. Het verbindt landen met kennis, ervaring en middelen om mensen te helpen een beter leven op te bouwen UNDP werkt in 170 ongeveer landen en gebieden, draagt bij tot de uitroeiing van armoede en gaat de ongelijkheden en uitsluiting tegen. Ze helpen de landen bij het ontwikkelen van hun ontwikkelingsbeleid, leiderschaps vaardigheden, partnerschap mogelijkheden, institutionele capaciteiten en het opbouwen van veerkracht om betere ontwikkelingsresultaten te bekomen. \autocite{DevelopmentProgram2020}
Hun werk is geconcentreerd op drie belangrijke aandachtsgebieden:

\begin{enumerate}
\item Duurzame ontwikkeling
\item Democratisch bestuur en vredesopbouw
\item Klimaat- en rampenbestendigheid
\end{enumerate}

 Jaarlijks brengt UNDP een Human Development Report uit. Dat concentreert zich op het mondiale debat over belangrijke ontwikkelingskwesties en biedt nieuwe meetinstrumenten, innovatieve analyses en vaak controversiële beleidsvoorstellen. 

\subsubsection{UNDP Strategic Plan}
Het Strategisch Plan (2018-2021) van UNDP is ontworpen om te reageren op de grote diversiteit van de landen die ze bedienen. Deze diversiteit wordt weerspiegeld in drie brede ontwikkelingscontexten: \autocite{DevelopmentProgram2020}

\begin{enumerate}
\item Roei armoede uit in al zijn vormen en dimensies
\item Versnel structurele transformaties
\item Bouw veerkracht op tegen schokken en crises
\end{enumerate}

Om deze brede doelen te bereiken heeft UNDP een reeks benaderingen opgesteld die ze hun "Signature Solutions" noemen:

\begin{enumerate}
	\item Mensen uit armoede houden
	\item Een bestuur een vreedzame, rechtvaardige en inclusieve samenlevingen
	\item Crisispreventie en verhoogde veerkracht
	\item milieu : natuurgebaseerde oplossingen voor ontwikkeling
	\item Schone, betaalbare energie
	\item Empowerment van vrouwen en gender gelijkheid
\end{enumerate}

\subsubsection{UN Capital Development Fund}
UNDP beheert ook het UN Capital Development Fund, dat ontwikkelingslanden helpt hun economie te laten groeien door bestaande bronnen van kapitaalhulp aan te vullen door middel van subsidies en leningen; en VN-vrijwilligers , die meer dan 6.500 vrijwilligers uit 160 landen vertegenwoordigen en 38 VN-partners ondersteunen ter ondersteuning van vrede, veiligheid, mensenrechten, humanitaire hulpverlening en ontwikkeling via vrijwilligerswerk wereldwijd.

\subsubsection{Sustainable Development Goals}
In september 2015 hebben wereldleiders de Agenda 2030 voor duurzame ontwikkeling aangenomen om een einde te maken aan armoede, de planeet te beschermen en ervoor te zorgen dat alle mensen vrede en welvaart genieten. UNDP werkt aan het versterken van nieuwe kaders voor ontwikkeling, rampenrisicovermindering en klimaatverandering. Wij ondersteunen de inspanningen van landen om de duurzame ontwikkelingsdoelen van mondiale doelen te bereiken, die de globale ontwikkelingsplannen tot 2030 zullen sturen.
 \autocite{DevelopmentProgram2020}
 
 \subsection{Peru als ontwikkelingsland}
 
 
\section{Het peruviaanse onderwijssysteem}
https://www.nuffic.nl/publicaties/onderwijssysteem-peru/

\section{ICT in peru}
\subsection{Digital gab}
\subsection{projecten in het verleden/organisaties ict hulp?}

